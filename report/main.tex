\documentclass[12pt, a4paper]{report}
\usepackage{geometry}
\geometry{margin=1in}  % Academic-standard margins
\usepackage{graphicx}   % For including images
\usepackage{amsmath, amsfonts}  % Math packages
\usepackage{hyperref}   % For hyperlinks
\usepackage{titlesec}   % Custom section titles
\usepackage{fancyhdr}   % Headers and footers
\usepackage{biblatex}   % For bibliography management
\addbibresource{references.bib}  % Reference file for citations

% Set up hyperref colors
\hypersetup{
    colorlinks = true,
    linkcolor = {blue!60!black},
    urlcolor = {blue!80!black},
    citecolor = {green!50!black}
}

% Formatting for chapters and sections
\titleformat{\chapter}[display]
  {\bfseries\Huge}{\chaptername~\thechapter}{0.5em}{}
\titleformat{\section}[block]{\Large\bfseries}{\thesection}{0.5em}{}

% Header and footer settings
\pagestyle{fancy}
\fancyhf{}
\fancyhead[L]{\leftmark}
\fancyhead[R]{\thepage}
\fancyfoot[C]{Confidential - Not for Distribution}

% Title and author
\title{\textbf{Title of the Academic Report} \\ \large Subtitle (if any)}
\author{Author Name \\ \textit{Department of XYZ, University Name}}
\date{\today}

\begin{document}

% Cover Page
\maketitle
\thispagestyle{empty}  % Suppress headers on title page

% Abstract
\begin{abstract}
\noindent This abstract summarizes the purpose, methods, and main findings of the report. It should give readers a brief overview of the report’s contents and significance, typically in around 150-200 words.
\end{abstract}

% Table of Contents
\clearpage
\tableofcontents
\thispagestyle{empty}  % No headers on TOC page
\newpage

% List of Figures and Tables
\listoffigures
\listoftables
\newpage

% Chapter 1 - Introduction
\chapter{Introduction}
\section{Background and Context}
Explain the background and broader context of the study. This section should frame the importance of the research.

\section{Objectives and Research Questions}
Clearly define the objectives and research questions addressed in the report.

\section{Significance of the Study}
Describe the potential contributions of this study to the field.

% Chapter 2 - Literature Review
\chapter{Literature Review}
Summarize key studies, theories, and findings relevant to the research.
\section{Previous Studies}
Review notable previous work, summarizing methodologies and findings.

\section{Research Gaps}
Identify gaps or limitations in existing literature that this study aims to address.

% Chapter 3 - Methodology
\chapter{Methodology}
Detail the research design, data collection methods, and analysis techniques.
\section{Research Design}
Provide an overview of the research framework and structure.

\section{Data Collection}
Describe the data sources, participants (if applicable), and collection methods.

\section{Data Analysis}
Outline the analysis methods, software, and statistical techniques used.

% Chapter 4 - Results
\chapter{Results}
Present the research findings clearly and logically.
\section{Descriptive Statistics}
Provide descriptive statistics and visualizations (e.g., graphs, tables).

\section{Key Findings}
Highlight key findings related to the research questions.

% Chapter 5 - Discussion
\chapter{Discussion}
Interpret the findings and relate them to the research objectives.
\section{Interpretation of Findings}
Analyze and interpret the significance of the results.

\section{Comparison with Previous Studies}
Discuss how these findings align or contrast with previous research.

% Chapter 6 - Conclusion
\chapter{Conclusion}
Summarize the study’s contributions, limitations, and recommendations for future research.
\section{Summary of Findings}
Briefly summarize the main findings and their implications.

\section{Limitations and Future Work}
Discuss the limitations of the study and suggest areas for further research.

% References
\chapter*{References}
\addcontentsline{toc}{chapter}{References}
\printbibliography

% Appendices
\appendix
\chapter{Appendix}
Include supplementary data, additional charts, or related materials here.

\end{document}
