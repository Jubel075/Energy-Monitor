\section{Introduction}

This project aims to develop and deploy an Energy Monitor system, designed to provide real-time energy consumption data for a residential setting. The core component of the system is an Arduino-based microcontroller, responsible for collecting and processing energy consumption data from various sensors. This data is then transmitted to a web server, allowing for remote monitoring and analysis. 

The web application, built using Python and Flask, provides a user-friendly interface to visualize energy consumption patterns, identify potential areas for energy savings, and generate insightful reports. This project leverages the power of open-source technologies to create a cost-effective and customizable solution for energy monitoring.

Key Features:
\begin{itemize}
    \item Real-time monitoring: The system provides up-to-date energy consumption data.
    \item Remote access: Users can access and analyze data from anywhere with an internet connection.
    \item Data visualization: The web application offers various visualization tools for easy data interpretation.
    \item Energy efficiency insights: The system helps identify areas for energy savings and optimization.
\end{itemize}

This document will detail the system's architecture, hardware implementation, software development, and deployment process. It will also include a comprehensive evaluation of the system's performance and potential future enhancements.