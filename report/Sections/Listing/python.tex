% Define the RGB colors using colortbl
\definecolor{RoyalBlue}{rgb}{0.254, 0.412, 0.882}    % RoyalBlue: Keywords
\definecolor{ForestGreen}{rgb}{0.133, 0.545, 0.133}   % ForestGreen: Comments
\definecolor{DarkOrange}{rgb}{1.000, 0.549, 0.000}    % DarkOrange: Strings
\definecolor{LightGray}{rgb}{0.929, 0.929, 0.929}     % LightGray: Background color

\lstdefinestyle{Python}{
  language=Python,
  basicstyle=\ttfamily\small,
  breaklines=true,
  numbers=left,
  numberstyle=\tiny\color{gray},
  keywordstyle=\bfseries\color{RoyalBlue},         % Keywords in RoyalBlue
  commentstyle=\itshape\color{ForestGreen},        % Comments in ForestGreen
  stringstyle=\color{DarkOrange},                  % Strings in DarkOrange
  backgroundcolor=\color{white},                   % Light background for readability
  frame=single,
  rulecolor=\color{black},
  upquote=true,
  morekeywords={EnergyMonitor, current, voltage, calcIrms},  % Add Arduino-specific keywords
}

\subsection{Backend Code (app.py)}

The main backend code in \texttt{app.py} initializes the Flask app, loads environment variables, connects to the PostgreSQL database, and defines routes for data retrieval and front-end rendering.

\subsubsection{Importing Libraries and Setting up Environment Variables}
\begin{lstlisting}[style=Python, caption={Importing Libraries and Setting up Environment Variables}]
from flask import Flask, render_template, jsonify, request
import pandas as pd
import plotly.graph_objects as go
from dotenv import load_dotenv
import os
import psycopg2
from psycopg2 import OperationalError

app = Flask(__name__)
load_dotenv()
DATABASE_URL = os.getenv("DATABASE_URL")
\end{lstlisting}

\subsubsection{Data Retrieval and Processing Function}
\begin{lstlisting}[style=Python, caption=app.py - Data Retrieval, frame=single]
def get_data(tab, group_by=None, selected_month=None, selected_day=None):
    try:
        conn = psycopg2.connect(DATABASE_URL)
        query = "SELECT date, irms, energy_usage, kwh FROM energydata"
        df = pd.read_sql_query(query, conn)
        conn.close()
    except OperationalError as e:
        print("Error connecting to the database:", e)
        return pd.DataFrame()

    df['date'] = pd.to_datetime(df['date'], format='%Y-%m-%d %H:%M:%S', errors='coerce')
    df['year'] = df['date'].dt.year
    df['month'] = df['date'].dt.month_name()
    df['day'] = df['date'].dt.day
    df['hour'] = df['date'].dt.hour
    df = df.sort_values(by='date')

    if selected_month:
        df = df[df['month'] == selected_month].sort_values(by=['day', 'date'])
    if selected_day:
        df = df[df['day'] == int(selected_day)].sort_values(by=['hour', 'date'])

    return df
\end{lstlisting}

\subsubsection{Main Route for the Application}
\begin{lstlisting}[style=Python, caption=app.py - Main Route, frame=single]
@app.route('/')
def index():
    selected_month = request.args.get('month')
    selected_day = request.args.get('day')
    selected_tab = request.args.get('tab', 'irms')
    df = get_data(tab=selected_tab, selected_month=selected_month, selected_day=selected_day)

    fig = go.Figure()
    if selected_tab == 'irms':
        fig.add_trace(go.Scatter(x=df['date'], y=df['irms'], mode='lines', name='Irms'))
        fig.update_layout(title="Current (Irms)", xaxis_title="Date", yaxis_title="Current (A)")
    elif selected_tab == 'energy_usage':
        fig.add_trace(go.Scatter(x=df['date'], y=df['energy_usage'], mode='lines', name='Energy Usage'))
        fig.update_layout(title="Energy Usage (Ws)", xaxis_title="Date", yaxis_title="Energy (Ws)")
    else:
        fig.add_trace(go.Scatter(x=df['date'], y=df['kwh'], mode='lines', name='kWh'))
        fig.update_layout(title="Energy Consumption (kWh)", xaxis_title="Date", yaxis_title="kWh")

    graph_html = fig.to_html(full_html=False)
    month_options = [{'value': month, 'label': month} for month in df['month'].unique()]
    day_options = df['day'].unique().tolist()

    return render_template('index.html', graph_html=graph_html,
                           selected_month=selected_month, selected_day=selected_day,
                           selected_tab=selected_tab, month_options=month_options,
                           day_options=day_options)
\end{lstlisting}

\subsubsection{Route for Fetching Day Options}
\begin{lstlisting}[style=Python, caption=app.py - Fetch Days, frame=single]
@app.route('/days', methods=['GET'])
def get_days():
    selected_month = request.args.get('month')
    df = get_data(tab='irms', selected_month=selected_month)
    days = df['day'].unique().tolist()
    return jsonify(days)
\end{lstlisting}

\subsubsection{Running the Flask Application}
\begin{lstlisting}[style=Python, caption=app.py - Run Flask App, frame=single]
if __name__ == '__main__':
    app.run(debug=True)
\end{lstlisting}