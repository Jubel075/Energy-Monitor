% Define custom colors
\definecolor{ArduinoBlue}{rgb}{0.0, 0.0, 1.0}
\definecolor{ArduinoGreen}{rgb}{0.0, 0.5, 0.0}
\definecolor{ArduinoRed}{rgb}{0.85, 0.0, 0.0}
\definecolor{ArduinoGray}{rgb}{0.5, 0.5, 0.5} 

% Define Arduino style
\lstdefinestyle{Arduino}{
  language=C++,
  backgroundcolor=\color{white},                % White background
  basicstyle=\ttfamily\small,                   % Use monospace font (default Arduino IDE font)
  keywordstyle=\color{ArduinoBlue}\bfseries,    % Blue color for keywords (e.g., void, int)
  commentstyle=\color{ArduinoGreen}\itshape,    % Green color for comments (e.g., // comment)
  stringstyle=\color{ArduinoRed} \itshape,      % Dark red color for strings
  numberstyle=\tiny\color{ArduinoGray},         % Small gray line numbers
  numbersep=5pt,                                % Distance between numbers and code
  frame=single,                                 % Border around code
  breaklines=true,                              % Wrap long lines
  showstringspaces=false,                       % Do not show spaces in strings
  captionpos=b,                                 % Place caption at the bottom
  numbers=left,                                 % Show line numbers
  stepnumber=1,                                 % Step through line numbers
  morekeywords={void, int, float, double, long, char, bool, for, while, if, else}, % Common Arduino keywords
  escapeinside={\%*}{*)},                       % Allow LaTeX commands inside listings
  literate={-}{-}{1},                            % Fix the minus signs in code
  identifierstyle=\color{ArduinoGray},          % Color for identifiers (variables, functions)
}

\subsection{Arduino Code Breakdown}

\subsubsection{Initializing Libraries and Variables}
\begin{lstlisting}[style=Arduino]
#include "EmonLib.h"
EnergyMonitor emon1;  // Create an instance

// Variables for measurements
double ACV;
double Irms;
double Power;
double Energy = 0.0;
double totalEnergy = 0.0;
double kWh = 0.0;
\end{lstlisting}

\subsubsection{Setup Function}
\begin{lstlisting}[style=Arduino]
void setup() {
  Serial.begin(9600);  
  emon1.current(A0, 20.85);  
  emon1.voltage(A2, 171.36, 1.7);  
}
\end{lstlisting}

\subsubsection{Main Loop for Measurements}
\begin{lstlisting}[style=Arduino]
void loop() {
  double Irms = emon1.calcIrms(1480); 
  double ACV = 127.0;  // Assume constant AC voltage of 127V
  double Power = Irms * ACV; 
  
  // Print the power to the serial monitor
  Serial.print("Power: ");
  Serial.println(Power);
}
\end{lstlisting}

\subsubsection{Energy Calculation and Printing Data}
\begin{lstlisting}[style=Arduino]
  // Calculate energy in watt-seconds for the 10-second interval
  Energy = Power * (MEASURE_INTERVAL / 1000.0);   
  totalEnergy += Energy;                          

  // Calculate kWh from total energy (in watt-seconds)
  kWh = totalEnergy / 3600000.0;  

  // Print data in CSV format for Python parsing
  printData();
\end{lstlisting}

\subsubsection{Simulating Time and Printing Data}
\begin{lstlisting}[style=Arduino]
void printData() {
  // Simulate incrementing time (1 measurement every 10 seconds)
  seconds += 10; 

  // Adjust minutes and hours based on seconds overflow
  if (seconds >= 60) {
    seconds -= 60;
    minutes++;
  }
  if (minutes >= 60) {
    minutes -= 60;
    hours++;
  }
  if (hours >= 24) {
    hours -= 24;
    day++;
  }

  // Format date and time string
  char dateBuffer[20];
  sprintf(dateBuffer, "%04d-%02d-%02d %02d:%02d:%02d", year, month, day, hours, minutes, seconds);

  // Print the data row with date and time
  Serial.print(dateBuffer);
  Serial.print(", ");
  Serial.print(Irms);
  Serial.print(", ");
  Serial.print(Energy);
  Serial.print(", ");
  Serial.println(kWh);
}
\end{lstlisting}